\section{Teoria}
\subsection{Opis Problemu}
Odległość edycyjna, znana także jako odległość Levenshteina, to popularna metoda do analizy i korekty tekstu, opracowana przez Vladimira Levenshteina w 1965 roku\cite{levenshtein1966}. Jest to uogólnienie odległości Hamminga, które mierzy odmienność między dwoma skończonymi ciągami znaków.

\vspace{1em}

Odległość ta jest zdefiniowana jako minimalna liczba prostych operacji potrzebnych do przekształcenia jednego ciągu w drugi. Dozwolone operacje to:

\begin{itemize}
    \item Zamiana znaku - zmiana jednego znaku na inny.
    \item Usunięcie znaku - usunięcie jednego znaku z ciągu.
    \item Dodanie znaku - wstawienie nowego znaku do ciągu.
\end{itemize}

\vspace{1em}

Każda z tych operacji ma taką samą wagę. Odległość Levenshteina jest miarą metryczną, co oznacza, że spełnia warunki metryki w przestrzeni ciągów znaków. Znajduje zastosowanie w różnych dziedzinach, takich jak rozpoznawanie mowy, analiza DNA, systemy antyplagiatowe i korekta pisowni.

\subsection{Rekurencyjna definicja}
Odległość Levenshteina można sformułować rekurencyjnie, co pozwala na efektywne obliczanie tej miary za pomocą programowania dynamicznego. Rekurencyjna definicja odległości edycyjnej \(d(x_1, x_2)\) między łańcuchami \(x_1\) i \(x_2\) jest następująca:

\vspace{1em}

Warunki brzegowe:
\begin{itemize}
    \item \(d(\epsilon, \epsilon) = 0\): Przekształcenie pustego łańcucha w pusty nie wymaga operacji.
    \item \(d(s, \epsilon) = |s|\): Przekształcenie łańcucha \(s\) w pusty wymaga usunięcia wszystkich znaków.
    \item \(d(\epsilon, s) = |s|\): Przekształcenie pustego łańcucha w \(s\) wymaga wstawienia wszystkich znaków.
\end{itemize}

Definicja rekurencyjna:
\[
    d(s_1z_1, s_2z_2) = \min 
    \begin{cases} 
        d(s_1, s_2) + \chi(z_1 \neq z_2), & \text{(zamiana)} \\ 
        d(s_1z_1, s_2) + 1, & \text{(usunięcie)} \\ 
        d(s_1, s_2z_2) + 1, & \text{(wstawienie)} 
    \end{cases}
\]
, gdzie \(\chi(z_1 \neq z_2)\) wynosi 0, jeśli znaki są takie same, i 1, jeśli są różne.