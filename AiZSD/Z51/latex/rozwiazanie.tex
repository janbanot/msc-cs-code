\section{Rozwiązanie}
\subsection{Struktura optymalnego rozwiązania}
Programowanie dynamiczne opiera się na podziale problemu na mniejsze, łatwiejsze do rozwiązania podproblemy oraz wykorzystaniu wyników tych podproblemów do budowy rozwiązania dla całości. W kontekście odległości edycyjnej, kluczową strukturą jest macierz kosztów, która systematycznie przechowuje wyniki obliczeń dla prefiksów obu łańcuchów.

\subsubsection{Główne etapy}

\begin{enumerate}
    \item \textbf{Podział na podproblemy} \\
    Odległość edycyjna pomiędzy całymi łańcuchami $s_1$ i $s_2$ jest rozbijana na mniejsze problemy, takie jak odległość między prefiksami tych łańcuchów. Prefiks oznacza dowolny początkowy fragment łańcucha, np. dla $s_1 = \text{``kot''}$, prefiksami są: $\text{``''}$, $\text{``k''}$, $\text{``ko''}$, $\text{``kot''}$.
    
    \item \textbf{Zasada optymalności} \\
    Minimalny koszt przekształcenia prefiksu $s_1[0..i]$ w $s_2[0..j]$ zależy wyłącznie od kosztów trzech możliwych operacji: zamiany, usunięcia lub wstawienia, zastosowanych do krótszych prefiksów. To pozwala budować rozwiązanie krok po kroku, bazując na wynikach mniejszych podproblemów.

    \item \textbf{Macierz kosztów} \\
    Tworzona jest dwuwymiarowa macierz $D$, gdzie $D[i][j]$ oznacza minimalny koszt przekształcenia prefiksu $s_1[0..i]$ w $s_2[0..j]$. Algorytm uzupełnia tę macierz od warunków brzegowych do wartości końcowej, tj. $D[m][n]$, gdzie $m = |s_1|$, $n = |s_2|$.
\end{enumerate}

\subsubsection{Przykład struktury}

Dla łańcuchów $s_1 = \text{``kot''}$ i $s_2 = \text{``młot''}$, proces budowy macierzy kosztów wygląda następująco:
\begin{itemize}
    \item Macierz $D$ inicjalizuje koszty przekształceń pustych prefiksów ($D[0][j]$, $D[i][0]$).
    \item Kolejne wartości w $D[i][j]$ są obliczane zgodnie z formułą rekurencyjną, wybierając minimalny koszt spośród dostępnych operacji.
\end{itemize}

W ten sposób struktura programu dynamicznego gwarantuje, że wszystkie podproblemy są rozwiązane efektywnie, unikając wielokrotnego obliczania tych samych wartości.

\subsection{Rekurencyjna funkcja kosztu}
\subsubsection{Definicja funkcji}
Niech \(d(i, j)\) oznacza minimalny koszt przekształcenia prefiksu \(s_1[0..i]\) w \(s_2[0..j]\).

\subsubsection{Warunki brzegowe}
\begin{itemize}
    \item \(d(0, 0) = 0\): Przekształcenie pustego łańcucha w pusty nie wymaga operacji.
    \item \(d(i, 0) = i\): Przekształcenie prefiksu \(s_1\) w pusty łańcuch wymaga \(i\) usunięć.
    \item \(d(0, j) = j\): Przekształcenie pustego łańcucha w prefiks \(s_2\) wymaga \(j\) wstawień.
\end{itemize}

\subsubsection{Rekurencyjne wyrażenie}
Dla \(i > 0\) i \(j > 0\):
\[
d(i, j) = \min \begin{cases} 
d(i-1, j) + 1, & \text{(usunięcie)} \\
d(i, j-1) + 1, & \text{(wstawienie)} \\
d(i-1, j-1) + \chi(s_1[i] \neq s_2[j]), & \text{(zamiana)}
\end{cases}
\] 
\label{eq:rekurencyjne-wyrazenie}
Gdzie \(\chi(s_1[i] \neq s_2[j])\) zwraca 0, jeśli znaki są takie same, lub 1, jeśli różne.

\subsubsection{Uzasadnienie}
Funkcja korzysta z wartości już obliczonych dla mniejszych podproblemów, co pozwala na efektywne obliczenie kosztu dla całych łańcuchów. Każda wartość w macierzy jest obliczana na podstawie wcześniej obliczonych wartości, co czyni algorytm efektywnym i pozwala na obliczenie minimalnej liczby operacji edycyjnych.

\subsection{Macierz kosztów}
Algorytm stosuje podejście rekurencyjne z memoizacją, co pozwala na dynamiczne obliczanie wartości tylko dla tych komórek, które są rzeczywiście potrzebne w trakcie działania programu. Wyniki są przechowywane w strukturze pamięci podręcznej (\textit{memo}), co zapobiega wielokrotnemu rozwiązywaniu tych samych podproblemów.

\begin{itemize}
    \item \textbf{Reprezentacja macierzy kosztów:} 
    Zamiast jawnej macierzy dwuwymiarowej $D[i][j]$, algorytm przechowuje wyniki w słowniku (\textit{memo}), gdzie kluczami są pary indeksów $(i, j)$, odpowiadające pozycjom w macierzy. Dla każdej pary $(i, j)$ wartość \textit{memo}$[(i, j)]$ oznacza minimalny koszt przekształcenia prefiksu $s_1[0..i]$ w $s_2[0..j]$.

    \item \textbf{Warunki brzegowe:} 
    Jeśli jeden z prefiksów jest pusty ($i = 0$ lub $j = 0$), koszt przekształcenia jest równy długości drugiego prefiksu:
    \[
    \textit{memo}[(i, 0)] = i, \quad \textit{memo}[(0, j)] = j.
    \]

    \item \textbf{Rekurencyjne wyrażenie:} 
    Koszt dla dowolnego $(i, j)$ jest obliczany rekurencyjnie na podstawie trzech możliwych operacji:
    \[
    \textit{memo}[(i, j)] = \min \begin{cases} 
    \textit{memo}[(i-1, j)] + 1 & \text{(usunięcie)} \\
    \textit{memo}[(i, j-1)] + 1 & \text{(wstawienie)} \\
    \textit{memo}[(i-1, j-1)] + \chi(s_1[i] \neq s_2[j]) & \text{(zamiana)} 
    \end{cases}
    \]
    gdzie $\chi(s_1[i] \neq s_2[j]) = 1$, jeśli znaki są różne, i $0$, jeśli są takie same.

    \item \textbf{Dynamiczne wypełnianie:} 
    Algorytm odwiedza komórki macierzy w kolejności zależnej od aktualnych potrzeb, rozpoczynając od $(m, n)$ (całych łańcuchów) i przechodząc rekurencyjnie do mniejszych podproblemów. Gdy wartość dla $(i, j)$ jest już obliczona, nie jest przeliczana ponownie, tylko odczytywana z \textit{memo}.
\end{itemize}

\subsubsection{Przykład ilustracyjny}
Rozważmy dwa łańcuchy “kot” i “młot”. Wypełniając macierz, możemy zobaczyć, jak wartości są obliczane krok po kroku. Każda komórka macierzy reprezentuje minimalny koszt przekształcenia odpowiednich prefiksów.

Oto tabela ilustrująca przykład dla wyrazów "kot" i "młot":
\begin{table}[H]
    \centering
    \begin{tabular}{c|c|c|c|c|c}
       &   & \text{m} & \text{\l} & \text{o} & \text{t} \\
    \hline
       & 0 & 1 & 2 & 3 & 4 \\
    \hline
    \text{k} & 1 & 1 & 2 & 3 & 4 \\
    \hline
    \text{o} & 2 & 2 & 2 & 2 & 3 \\
    \hline
    \text{t} & 3 & 3 & 3 & 3 & 2 \\
    \end{tabular}
    \caption{Tabela przedstawiająca minimalny koszt przekształcenia wyrazu "kot" w "młot".}
    \label{tab:kot-mlot}
\end{table}

\begin{itemize}
    \item Inicjalizacja - pierwszy wiersz i kolumna reprezentują przekształcenia pustego łańcucha w prefiksy drugiego łańcucha i odwrotnie.
    \item Obliczenia - każda komórka \(D[i][j]\) jest obliczana na podstawie sąsiednich komórek, zgodnie z rekurencyjną funkcją kosztu.
    \item Wynik końcowy - komórka \(D[3][4]\) (wartość 2) oznacza minimalny koszt przekształcenia "kot" w "młot".
\end{itemize}

\textit{Powyższa tabela ilustruje pełne wypełnienie macierzy kosztów. W podejściu z memoizacją obliczane są tylko te komórki, które są rzeczywiście potrzebne, a wyniki przechowywane w strukturze \textit{memo}.}