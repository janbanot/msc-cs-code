\section{Odpowiedź}
\subsection{Algorytm w pseudokodzie}
Poniższy fragment pseudokodu ilustruje, jak algorytm wypełnia macierz kosztów, korzystając z podejścia zstępującego z memoizacją, co pozwala na efektywne obliczanie odległości edycyjnej.

\begin{verbatim}
Funkcja OdległośćEdycyjna(s1, s2):
    Jeśli memo nie istnieje:
        // Struktura do przechowywania wyników podproblemów
        Utwórz pusty memo
    // Klucz identyfikuje unikalny podproblem
    Klucz = (długość(s1), długość(s2))
    Jeśli Klucz jest w memo:
        Zwróć memo[Klucz]
    Jeśli długość(s1) == 0 i długość(s2) == 0:
        // Brak operacji dla pustych łańcuchów
        Zwróć 0
    Jeśli długość(s1) == 0:
        // Wstawienie wszystkich znaków z s2
        Zwróć długość(s2)
    Jeśli długość(s2) == 0:
        // Usunięcie wszystkich znaków z s1
        Zwróć długość(s1)
    s1_prefix = s1 bez ostatniego znaku
    s2_prefix = s2 bez ostatniego znaku
    z1 = ostatni znak s1
    z2 = ostatni znak s2
    koszt_zamiany = 0 jeśli z1 == z2, w przeciwnym razie 1
    Wynik = min(
        // Zamiana
        OdległośćEdycyjna(s1_prefix, s2_prefix) + koszt_zamiany,
        // Wstawienie
        OdległośćEdycyjna(s1, s2_prefix) + 1,
        // Usunięcie
        OdległośćEdycyjna(s1_prefix, s2) + 1
    )
    memo[Klucz] = Wynik
    
    Zwróć Wynik  
\end{verbatim}

\subsection{Analiza złożoności}
\begin{itemize}
    \item Złożoność obliczeniowa - dzięki memoizacji, złożoność czasowa wynosi \(O(m \times n)\), gdzie \(m\) i \(n\) to długości porównywanych łańcuchów. Każdy podproblem jest rozwiązywany tylko raz.
    \item Złożoność pamięciowa - pamięć potrzebna do wykonania procesu memoizacji to również \(O(m \times n)\), ponieważ przechowujemy pośrednie wyniki dla wszystkich możliwych podproblemów.
\end{itemize}

\subsection{Implementacja rozwiązania w języku Python}
Poniżej znajduje się implementacja algorytmu obliczającego odległość edycyjną z wykorzystaniem memoizacji. Podejście to pozwala na efektywne obliczanie minimalnej liczby operacji potrzebnych do przekształcenia jednego łańcucha w drugi, unikając wielokrotnego rozwiązywania tych samych podproblemów.

\begin{verbatim}
def edit_distance(s1, s2, memo=None):
    if memo is None:
        memo = {}
    
    key = (len(s1), len(s2))
    # If result already calculated, return it
    if key in memo:
        return memo[key]
        
    # Base cases
    if len(s1) == 0:
        return len(s2)
    if len(s2) == 0:
        return len(s1)
        
    # Get prefixes and last characters
    s1_prefix = s1[:-1]
    s2_prefix = s2[:-1]
    z1 = s1[-1]
    z2 = s2[-1]

    # Calculate replacement cos
    rep_cost = 0 if z1 == z2 else 1
    
    result = min(
        edit_distance(s1_prefix, s2_prefix, memo) + repl_cost,
        edit_distance(s1, s2_prefix, memo) + 1,
        edit_distance(s1_prefix, s2, memo) + 1,
    )
    
    memo[key] = result
    return result
\end{verbatim}